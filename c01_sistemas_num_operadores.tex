\begin{frame} \frametitle{Sistemas de numeración} % uno

\begin{itemize}
\item Números Naturales: $\mathbb{N} = \{1, 2, 3, 4, ...\}$
\item Números Enteros: $\mathbb{E} = \{..., -3, -2, -1, 0, 1, 2, 3, 4, ...\}$
\item Números Racionales: $\mathbb{Q} = \{ \frac{p}{q} | p,q \in \mathbb{E}, \text{ con } q \neq 0 \}$
\item Números Irracionales: $\mathbb{I} = \{ ..., -\pi, \pi, e, \sqrt{2} ... \}$  
\item Números Reales: $\mathbb{R} = \mathbb{Q} \cup \mathbb{I} $
\item Números Complejos: $\mathbb{C} = \{ a+ ib = (a,b) | a, b \in \mathbb{R}, i=\sqrt{-1} \}$  
\end{itemize}

\end{frame}

\begin{frame}\frametitle{Operadores Aritméticos y de Comparación} % dos

\begin{itemize}
\item Operadores Aritméticos: +, -, *, /, \^.
\item Operadores Comparación: $ <, \leq, >, \geq, \neq $ 
\item Operadores Lógicos: \& (and), $|$ (or)
\item Valores Lógicos: TRUE (1), FALSE (0)
\end{itemize}


\end{frame}

\begin{frame}\frametitle{Algunas operaciones} % tres

$$
\begin{matrix}
\textbf{and} & \vline  TRUE  & FALSE \\
\hline
TRUE & \vline  TRUE &  FALSE \\
FALSE & \vline FALSE & FALSE
\end{matrix}
$$


$$
\begin{matrix}
\textbf{or} & \vline  TRUE  & FALSE \\
\hline
TRUE & \vline  TRUE &  TRUE \\
FALSE & \vline TRUE & FALSE
\end{matrix}
$$


\end{frame}

\begin{frame}\frametitle{Vectores y algunas operaciones} cuatro

Se define un vector como:

$$
\textbf{v} = (x_1, x_2, x_3, ..., x_n) \text{ donde } x_i \in \mathbb{R}
$$

Se define

$$
\mathbb{R}^n = \{(x_1, x_2, x_3, ..., x_n) | x_i \in \mathbb{R} \}
$$ 

\end{frame}

\begin{frame} % cinco

Sean $\textbf{v}_1 = (x_1, x_2, x_3, ..., x_n)$ y $\textbf{v}_2 = (y_1, y_2, y_3, ..., y_n)$  
dos vectores en $\mathbb{R}^n$ y $a \in \mathbb{R}$ es un numero llamado escalar, se define:

\begin{enumerate}
\item $\textbf{0} \in \mathbb{R}^n$ como \textbf{0} = (0,0, ..., 0).
\iten $\textbf{v}_1 + \textbf{v}_2 = (x_1+y_1, x_2+y_2, x_3+y_3, ..., x_n+y_n)$
\item $a * \textbf{v}_1 = a \textbf{v}_1 = a(x_1, x_2, x_3, ..., x_n) = (ax_1, ax_2, ax_3, ..., ax_n)$
\end{enumerate}

\end{frame}

\begin{frame}\frametitle{Matrices y sus Operaciones} % seis

$$
\textbf{A} = \begin{pmatrix}
a_{11} & a_{12} & a_{13} & ...  & a_{1n} \\
a_{21} & a_{22} & a_{23} & ...  & a_{2n} \\
... & ... & ... & ... & ... \\
a_{m1} & a_{m2} & a_{m3} & ...  & a_{mn} 
\end{pmatrix} = (a_{ij})_{i=1,...,m, j=1,...,n}
$$ donde $a_{ij} \in \mathbb{R}$

decimos que es de tamaño $m \times n$. Contiene m filas y n columnas.

\end{frame}

\begin{frame} % site

Dadas las matrices  $\textbf{A} = (x_{ij})_{i=1,...,m, j=1,...,n}$ y  $\textbf{B} = (y_{ij})_{i=1,...,m, j=1,...,n}$ y el escalar $a \in \mathbb{R}$ definimos las
siguientes operaciones:

\begin{enumerate}
\item $\textbf{0}*A = \textbf{0} (x_{ij}) = (0_{ij})_{i=1,...,m, j=1,...,n}$
\item $\textbf{A+B} = (x_{ij}) + (y_{ij}) = (x_{ij} + y_{ij})$
\item $\textbf{a}*A = \textbf{a} (x_{ij}) = (ax_{ij})$
\end{enumerate}

\end{frame}

\begin{frame} %ocho


\end{frame}

